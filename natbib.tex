%\ProvidesFile{natbib}
% [2010/09/13 8.31b (PWD, AO)]
% \changes{4.0}{1993 Aug 19}{First documented release}
% \changes{4.1}{1993 Oct 4}{Simplification of \cs{@citeapalk}}
% \changes{4.1a}{1993 Oct 14}{Add \texttt{rev} option for reversed comments
%                             in \cs{cite}}
% \changes{4.1b}{1993 Oct 18}{Add \cs{bibfont} to list definition =\cs{relax}}
% \changes{4.2}{1993 Oct 22}{Add coding for AGU, NLINPROC}
% \changes{4.2}{1993 Nov 20}{Add more coding for AGU}
% \changes{4.3a}{1994 Feb 24}{First additions for \LaTeXe}
% \changes{5.0}{1994 May 18}{Revised for \LaTeXe{} and 2.09}
% \changes{5.0}{1994 May 18}{Remove obsolete JGR, GRL coding}
% \changes{5.0}{1994 May 18}{Add \cs{citeauthor}, \cs{citeyear}}
% \changes{5.0}{1994 May 18}{Two optional texts for \cs{cite} so \texttt{rev}
%                            option obsolete}
% \changes{5.0}{1994 May 18}{\LaTeXe\ options to select punctuation}
% \changes{5.1}{1994 Jun 22}{Conform to first official release of \LaTeXe}
% \changes{5.1}{1994 Jun 22}{Separate \LaTeX\ and 2.09 files}
% \changes{5.1}{1994 Jun 22}{Put doc driver first}
% \changes{5.2}{1994 Aug 25}{Fix up 2.09 style to run in compatibility mode}
% \changes{5.2}{1994 Aug 25}{\cs{citeauthor}, \cs{citeyear} make BibTeX
%                             entry in aux file}
% \changes{5.2}{1994 Aug 25}{\cs{@citex} defined as in \LaTeXe}
% \changes{5.2}{1994 Aug 25}{Local config file \texttt{natbib.cfg} read in}
% \changes{5.3}{1994 Sep 13}{Add \cs{citefullauthor}, options \texttt{angle},
%                             \texttt{curly}}
% \changes{5.3}{1994 Sep 19}{Add star version of \cs{cite} for full authors}
% \changes{5.3}{1994 Sep 26}{Fix accents in citations with proper definition
%                             of \cs{protect}}
% \changes{5.4}{1994 Nov 24}{Add space in \cs{@citex} for text cites}
% \changes{5.4}{1994 Nov 24}{Replace \cs{if@tempswa} by \cs{ifNAT@swa}}
% \changes{5.4}{1994 Nov 24}{Add superscript citation type to \cs{bibpunct}}
% \changes{5.4}{1994 Nov 24}{Add \cs{@citesuper}, fix up bugs in superscripts}
% \changes{5.4}{1994 Nov 24}{Define \cs{@citexnum} as in \LaTeXe}
% \changes{5.4}{1995 Feb 03}{Add \cs{citestyle} same as \cs{bibstyle}}
% \changes{5.4}{1995 Feb 08}{For repeated years and authors, print just letter}
% \changes{5.5}{1995 Mar 13}{Add \cs{bibhang} and command space in
%      \cs{@cite}}
% \changes{5.5}{1995 Mar 16}{Add \cs{citealt} for citation with no
%      parentheses}
% \changes{5.5}{1995 Mar 24}{Reorganize internal commands, using \cs{NAT@}
%     prefixes}
% \changes{5.5}{1995 Mar 24}{Punctuation selection commands \cs{bibpunct},
%     \cs{citestyle} are now preamble only, whereas previously they had to
%     come after the preamble}
% \changes{5.5}{1995 May 14}{Change names of punctuation commands to
%     \cs{NAT@...}}
% \changes{6.0}{1995 Sep 4}{Allow numerical styles with author-year
%     \texttt{bst} files}
% \changes{6.0}{1995 Sep 21}{Add automatic indexing of citations}
% \changes{6.0}{1995 Sep 29}{Accommodate \texttt{index} package}
% \changes{6.1}{1995 Nov 22}{Fixed for \LaTeXe\ \texttt{1995/12/01}}
% \changes{6.1}{1995 Dec 4}{Make more robust against changes to internals}
% \changes{6.1a}{1995 Dec 19}{Fix test for changed citations}
% \changes{6.2}{1996 Jan 10}{Replace all \cs{uppercase}}
% \changes{6.2}{1996 Jan 11}{Add \cs{citet}}
% \changes{6.2}{1996 Feb 2}{Fix superscript size}
% \changes{6.2}{1996 Mar 05}{Add length \cs{bibsep} for linespacing between
%               references}
% \changes{6.2}{1996 Apr 15}{Fix clash with \texttt{amsart} and
%              \texttt{amsbook}}
% \changes{6.3}{1996 Jun 10}{Allow \texttt{showkeys} to be loaded first}
% \changes{6.3}{1996 Jun 17}{Fix punctuation for \texttt{plainnat}}
% \changes{6.3}{1996 Jun 17}{Suppress extra labels with numericals}
% \changes{6.4}{1996 Jun 18}{Provide \cs{bibname} and \cs{refname}}
% \changes{6.4}{1996 Jun 27}{Change \texttt{nlinproc} option to \texttt{egs}}
% \changes{6.4}{1996 Sep 1}{Make compatible with \texttt{chapterbib.sty}}
% \changes{6.4}{1996 Sep 12}{Fix spacing for superscripts}
% \changes{6.4}{1996 Sep 12}{Extra letter printed with \cs{citeyear}}
% \changes{6.4}{1996 Sep 13}{Add compression and sorting of numerical citations}
% \changes{6.4}{1996 Oct 2}{Make compatible with \texttt{hyperref.sty}}
% \changes{6.5}{1996 Dec 11}{KOMA script compatibility}
% \changes{6.5}{1997 Jan 10}{For EGS, no blank line between references}
% \changes{6.5}{1997 Jan 30}{Recode notes so they work with \cs{citet} and
%     \cs{citep}; change documentation to stress these commands over \cs{cite}}
% \changes{6.5}{1997 Feb 5}{Fix KOMA script properly}
% \changes{6.6}{1997 Apr 6}{Fix \cs{nocite} to work with \texttt{chapterbib}}
% \changes{6.6}{1997 May 26}{Let \cs{citealt} have full \cs{cite} syntax}
% \changes{6.6}{1997 Jun 4}{Add \cs{citeyearpar}}
% \changes{6.6}{1997 Jun 4}{Add sorting of author--year citations}
% \changes{6.6}{1997 Jun 12}{Add \cs{citealp} like \cs{citep} without parentheses}
% \changes{6.6}{1997 Jun 24}{Fix \texttt{showkeys} functionality}
% \changes{6.6}{1997 Jun 25}{Improve \texttt{hyperref} functionality}
% \changes{6.6}{1997 Jun 30}{Fix bug in \cs{NAT@citex}}
% \changes{6.7}{1997 Jul 14}{Add \texttt{longnamesfirst} option}
% \changes{6.7}{1997 Sep 12}{Fix interface with \texttt{showkeys}}
% \changes{6.7}{1997 Nov 10}{Fix interface with \texttt{babel}}
% \changes{6.7}{1997 Nov 11}{Add reference sheet extraction}
% \changes{6.8}{1997 Dec 1}{Permit \cs{@biblabel} to be user modified}
% \changes{6.8}{1998 Feb 19}{Fix hyperref bug in \cs{citep}}
% \changes{6.8a}{1998 Mar 6}{Correct wrong name for \texttt{longnamesfirst}
%    in the documentation}
% \changes{6.8a}{1998 May 14}{\cs{@biblabel} only redefinable for numerical
%    mode}
% \changes{6.8b}{1998 July 6}{Hyperref: to work with \texttt{chapterbib} and
%     breaks links in textual citations}
% \changes{6.8c}{1998 July 14}{Hyperref: remove opening brace and notes from
%     link}
% \changes{6.8d}{1998 Nov 23}{Fix bug in \texttt{super} option; suppress brackets}
% \changes{6.9}{1999 Jan 22}{Fix bug with \cs{makeindex}}
% \changes{6.9}{1999 Feb 23}{Update copyright notice}
% \changes{6.9}{1999 Mar 26}{Babel test made more general}
% \changes{7.0}{1999 May 7}{With empty year, act like \cs{citeauthor}}
% \changes{7.0}{1999 May 21}{Correct \cs{bibsection} for \texttt{amsbook}}
% \changes{7.0a}{2000 Jul 24}{Argument of \cs{citenumfont} always in braces}
% \changes{7.0b}{2002 Feb 27}{Fix bug with superscripts}
% \changes{7.1}{2003 June 6}{Add test for missing extra letter in multi-cites}
% \changes{7.2}{2006 Jan 12}{Allow \texttt{hyperref} with compression}
% \changes{7.2}{2006 Jan 12}{Add compression without sorting}
% \changes{7.2}{2006 Jan 12}{Add compatibility with \texttt{citeref} package}
% \changes{7.3}{2006 Mar 22}{Revise and update the manual documentation}
% \changes{7.4}{2006 Aug 18}{Add \cs{NAT@nmfmt} to names in \cs{NAT@test}}
% \changes{7.4a}{2006 Sep 6}{Fix bug with compress option and hyperref with dvips}
% \changes{7.5}{2007 Feb 2}{Add \cs{citenum} to print (non-superscripted) number}
% \changes{7.5}{2007 Feb 2}{Superscripted braces raised in textual citation too}
% \changes{7.5}{2007 Feb 2}{Change notes for textual numerical citations to be outside}
% \changes{8.0}{2007 Feb 2}{Allow changes in cite style, for chapterbib; remove \texttt{nopreonly} option}
% \changes{8.0}{2007 Feb 5}{Remove 2.09 support}
% \changes{8.0}{2007 Feb 5}{Remove subpackage option (and agu, egs options)}
% \changes{8.1}{2007 Oct 30}{Replace some \cs{edef}s with \cs{protected@edef}}
% \changes{8.2}{2008 Jan 10}{(AO)Introduce, for \texttt{merge} package option,
%    \cs{NAT@bibitem@first@sw} and \cs{NAT@bibitem@label@sw}}
% \changes{8.2}{2008 Jan 10}{(AO) The sort algorithm had not been stable; fixed.}%
% \changes{8.2}{2008 Jan 10}{(AO) Various changes to \texttt{NAT@sort@cites} and friends}%
% \changes{8.2}{2008 Jun 26}{(AO) Employ \cs{NAT@ifcat@num} to make the test on \cs{NAT@num}}%
% \changes{8.2}{2008 Jun 26}{(AO) Use \cs{NAT@space}, \cs{NAT@spacechar}, and \cs{NAT@penalty}
%     instead of explicit space and primitive penalty in \cs{NAT@idxtxt},
%     \cs{NAT@cite}, \cs{NAT@citenum}, \cs{NAT@citesuper}.}%
% \changes{8.2}{2008 Jun 28}{(AO) Introduce \cs{NAT@ifcat@num}, ensuring that the
%      \cs{catcode} of the underscore is 8 at the time;
%      use for tests on \cs{NAT@num} and \cs{NAT@last@num}.}%
% \changes{8.2}{2008 Jul 01}{(AO) Allow graceful transition away from the use of \texttt{natbib}}%
% \changes{8.2}{2008 Jul 01}{(AO) Assign \cs{NAT@ctype} with \cs{let} \cs{z@}:
%      allow for tighter \cs{ifnum} comparisons.}%
% \changes{8.2}{2008 Jul 01}{(AO) Give \cs{NAT@split} a good argument,
%       even when \cs{NAT@temp} is poorly constructed.}%
% \changes{8.2}{2008 Jul 01}{(AO) Provide a layer of abstraction to assignments
%       to \cs{@citea}, using \cs{NAT@def@citea} and \cs{NAT@reset@citea}.}%
% \changes{8.2}{2008 Jul 01}{(AO) With our abstraction of \cs{@citea};
%       we use wrappers \cs{NAT@def@citea@box} and \cs{NAT@hyper@citea}. Also \cs{NAT@penalty} in place of an explicit \cs{penalty}.}%
% \changes{8.2}{2008 Jul 01}{(AO) Use \cs{NAT@hyper@} and \cs{NAT@hyper@citea}
%       to wrap up the hyperlink calls in \cs{NAT@citexnum}, \cs{NAT@citex}, .}%
% \changes{8.2}{2008 Jul 02}{(AO) Add \cs{@ifxundefined}, a tool for testing a \cs{TeX} macro,
%       and \cs{@ifnum}, a tool for testing numbers, like \cs{ifnum}.}%
% \changes{8.2}{2008 Jul 02}{(AO) Employ \cs{@ifxundefined}, instead of \cs{@ifundefined},
%       for use on \cs{@cprwrite}, \cs{@indexfile}, \cs{bbl@redefine}, \cs{chapter},
%       \cs{NAT@sectionbib}, and \cs{bib@heading}: the \cs{csname} being tested ought not to be altered as a side effect.}%
% \changes{8.2}{2008 Jul 02}{(AO) The value of \cs{NAT@sort} and \cs{NAT@cmprs}
%       will be \cs{z@} or \cs{@ne}: something created by \cs{chardef}. This works better in \cs{ifnum} tests.}%
% \changes{8.2}{2008 Jul 12}{(AO) Do not write .aux file entries in \cs{NAT@citexnum}
%       or \cs{NAT@citex}; do this in \cs{NAT@sort@cites} where the keys will be in the user's order.}
% \changes{8.2}{2008 Jul 13}{(AO) Allow optional arguments preceding cite key.}
% \changes{8.2}{2008 Jul 13}{(AO) Commented-out code has been made into a comment.}
% \changes{8.2}{2008 Jul 13}{(AO) Protect against case where \cs{NAT@num} is
%       defined to be the same as \cs{@empty} (malformed \texttt{.aux} file).}
% \changes{8.3}{2008 Aug 28}{Refine the compatibility with babel}
% \changes{8.3}{2008 Sep 23}{Fix problem with spacing and raised commas for multiple superscripts}
% \changes{8.3}{2008 Nov 5}{Repair a number of bugs from AO's coding}
% \changes{8.3}{2008 Dec 8}{Put all comments inside braces with \cs{citet},
%    even for numerical (previously only for author--year)}
% \changes{8.3}{2009 Feb 2}{Include AO's last delta corrections}
% \changes{8.31}{2009 Jul 16}{Fix so when merge is 0, old syntax valid; backward compatibility}
% \changes{8.31a}{2009 Nov 07}{(AO) Fix punctuation bugs in merge code}
% \changes{8.31b}{2010 Sep 13}{Remove definitions of \cs{bibAnnote} and \cs{bibAnnoteFile} when merging not selected}
%
% \CheckSum{2884}
% \CharacterTable
%  {Upper-case    \A\B\C\D\E\F\G\H\I\J\K\L\M\N\O\P\Q\R\S\T\U\V\W\X\Y\Z
%   Lower-case    \a\b\c\d\e\f\g\h\i\j\k\l\m\n\o\p\q\r\s\t\u\v\w\x\y\z
%   Digits        \0\1\2\3\4\5\6\7\8\9
%   Exclamation   \!     Double quote  \"     Hash (number) \#
%   Dollar        \$     Percent       \%     Ampersand     \&
%   Acute accent  \'     Left paren    \(     Right paren   \)
%   Asterisk      \*     Plus          \+     Comma         \,
%   Minus         \-     Point         \.     Solidus       \/
%   Colon         \:     Semicolon     \;     Less than     \<
%   Equals        \=     Greater than  \>     Question mark \?
%   Commercial at \@     Left bracket  \[     Backslash     \\
%   Right bracket \]     Circumflex    \^     Underscore    \_
%   Grave accent  \`     Left brace    \{     Vertical bar  \|
%   Right brace   \}     Tilde         \~}
%
% \iffalse
%<*driver>
\documentclass{ltxdoc}
%<driver>%\documentclass[twoside]{ltxdoc}
%<driver>%\documentclass[a4paper]{ltxdoc}
%<driver>%\documentclass[twoside,a4paper]{ltxdoc}
\raggedbottom

 %** To include the detailed explanation of the coding, comment out
 %**   the next line
\OnlyDescription

 %** To produce a command index: add the following line for one run,
 %**   then run  makeindex -s gind.ist natbib
 %**   and reprocess, with or without this line (much faster without)
%<driver>% \EnableCrossrefs\CodelineIndex

 %** To produce a change history: add the following line for one run,
 %**   then run  makeindex -s gglo.ist -o natbib.gls natbib.glo
 %**   and reprocess, with or without this line (faster without)
%<driver>% \RecordChanges

\DisableCrossrefs %May stay; zapped by \EnableCrossrefs
\CodelineNumbered %May stay

\begin{document}
   \DocInput{natbib.tex}
\end{document}
%</driver>
%\fi
%
\DoNotIndex{\begin,\CodelineIndex,\CodelineNumbered,\def,\DisableCrossrefs}
\DoNotIndex{\DocInput,\documentclass,\EnableCrossrefs,\end,\GetFileInfo}
\DoNotIndex{\NeedsTeXFormat,\OnlyDescription,\RecordChanges,\usepackage}
\DoNotIndex{\ProvidesClass,\ProvidesPackage,\ProvidesFile,\RequirePackage}
\DoNotIndex{\LoadClass,\PassOptionsToClass,\PassOptionsToPackage}
\DoNotIndex{\DeclareOption,\CurrentOption,\ProcessOptions,\ExecuteOptions}
\DoNotIndex{\AtEndOfClass,\AtEndOfPackage,\AtBeginDocument,\AtEndDocument}
\DoNotIndex{\InputIfFileExists,\IfFileExists,\ClassError,\PackageError}
\DoNotIndex{\ClassWarning,\PackageWarning,\ClassWarningNoLine}
\DoNotIndex{\PackageWarningNoLine,\ClassInfo,\PackageInfo,\MessageBreak}
\DoNotIndex{\space,\protect,\DeclareRobustCommand,\CheckCommand}
\DoNotIndex{\newcommand,\renewcommand,\providecommand,\newenvironment}
\DoNotIndex{\renewenvironment,\newif,\newlength,\newcounter,\setlength}
\DoNotIndex{\setcounter,\if,\ifx,\ifcase,\ifnum,\ifdim,\else,\fi}
\DoNotIndex{\texttt,\textbf,\textrm,\textsl,\textsc,\reset@font}
\DoNotIndex{\textup,\textit,\textmd,\textsf,\emph,\futurelet}
\DoNotIndex{\ttfamily,\rmfamily,\sffamily,\mdseries,\bfseries,\upshape}
\DoNotIndex{\slshape,\scshape,\itshape,\em,\LaTeX,\LaTeXe}
\DoNotIndex{\filename,\fileversion,\filedate,\let,\makeindex}
\DoNotIndex{\@auxout,\@for,\@gobble,\@ifnextchar,\@m,\@mkboth,\@nil}
\DoNotIndex{\@noitemerr,\@tempa,\@tempswafalse,\@tempswatrue,\@warning}
\DoNotIndex{\advance,\arabic,\AtBeginDocument,\bf,\bibname,\chapter}
\DoNotIndex{\citation,\clubpenalty,\CodelineNumbered,\csname}
\DoNotIndex{\DisableCrossrefs,\do,\edef,\else,\endcsname,\endlist}
\DoNotIndex{\expandafter,\fi,\gdef,\global,\hbox,\hfill,\hskip,\hspace}
\DoNotIndex{\if,\if@filesw,\if@tempswa,\ifx,\immediate,\itemindent,\labelsep}
\DoNotIndex{\labelwidth,\lastskip,\leftmargin,\list,\mbox,\newblock}
\DoNotIndex{\newpage,\p@enumiv,\parindent,\penalty,\refname}
\DoNotIndex{\relax,\section,\settowidth,\sfcode,\sloppy,\small,\string}
\DoNotIndex{\theenumiv,\thepage,\unskip,\uppercase,\usecounter,\vskip}
\DoNotIndex{\widowpenalty,\write,\xdef,\z@,\catcode,\ifnum,\the}
\DoNotIndex{\.,\@empty,\@ifundefined,\@latex@warning,\@minus,\@plus,\ }
\DoNotIndex{\document,\@namedef,\@listi,\markboth,\or,\p@}
\DoNotIndex{\listparindent,\noexpand,\par,\parsep,\pb,\pbf,\pbfseries}
\DoNotIndex{\pc,\pd,\pem,\pit,\pitshape,\pmdseries,\prm,\prmfamily,\psc}
\DoNotIndex{\pscshape,\psf,\psffamily,\psl,\pslshape,\ptt,\pttfamily}
\DoNotIndex{\pupshape,\@iden,\@firstofone,\@unexpandable@protect}
\DoNotIndex{\&,\{,\},\bibitem,\bibindent,\if@draft,\typeout}
\DoNotIndex{\@ifclassloaded,\@ifstar,\@onlypreamble,\@preamblecmds}
\DoNotIndex{\addtolength,\endinput,\@bsphack,\begingroup,\@wrindex}
\DoNotIndex{\@listctr,\bibname,\enddocument,\hfil,\ignorespaces,\item}
\DoNotIndex{\NAT@temp,\refname,\stepcounter,\@ifpackageloaded}
\DoNotIndex{\@gobbletwo,\index,\itemsep,\markright,\scriptsize}
\DoNotIndex{\textsuperscript,\@undefined}
\DoNotIndex{\@celt,\@cite@list,\@citea,\@citeb,\@compress@cite}
\DoNotIndex{\@bsphack,\@esphack,\@h@ld,\@make@cite@list,\@ne}
\DoNotIndex{\@sort@celt,\@tempcnta,\@tempcntb,\delimiter,\endgroup}
\DoNotIndex{\ifcat,\m@ne,\number,\kern,\protected@edef}
\DoNotIndex{\@markboth,\@mkboth,\nobreak,\SK@,\SK@@citex}
\DoNotIndex{\SK@@label,\SK@@ref,SK@def,\SK@lbibitem}
\DoNotIndex{\@nameuse,\@extra@b@citeb,\@tempb,\@tempc}

\setcounter{IndexColumns}{2}
\setlength{\IndexMin}{10cm}
\setcounter{StandardModuleDepth}{1}

\hyphenation{par-en-the-ti-cal}

\GetFileInfo{natbib}

\title{{\bfseries Natural Sciences Citations and References}\\
        (Author--Year and Numerical Schemes)}

\author{Patrick W. Daly\thanks{%
  The code for merged numerical bibliography entries, emulating the \texttt{mcite} package,
     has been commissioned by the American Physical Society and provided by Arthur Ogawa}}

\date{This paper describes package \texttt{\filename}\\
      version \fileversion{} from \filedate.}

\maketitle

\pagestyle{myheadings}
\markboth{P. W. Daly}{NATURAL SCIENCES CITATIONS AND REFERENCES}

\newcommand{\btx}{\textsc{Bib}\TeX}
\newcommand{\thestyle}{\texttt{\filename}}


^^A In order to keep all marginal notes on the one (left) side:
^^A (otherwise they switch sides disasterously with twoside option)
\makeatletter \@mparswitchfalse \makeatother

\begin{abstract}

The \thestyle\ package is a reimplementation of the \LaTeX\ |\cite| command,
to work with both author--year and numerical citations. It is compatible with
the standard bibliographic style files, such as \texttt{plain.bst}, as well as
with those for \texttt{harvard}, \texttt{apalike}, \texttt{chicago},
\texttt{astron}, \texttt{authordate}, and of course \thestyle.


In contrast to the packages listed above, the \thestyle{} package supports not
only the various author--year bibliography styles, but also those for standard
numerical citations. In fact, it can also produce numerical citations even
with an author--year bibliographic style, something that permits easy
switching between the two citation modes. To this end, replacements for the
standard \LaTeX\ \texttt{.bst} files are also provided.

It is possible to define the citation \emph{style} (type of brackets and
punctuation between citations) and even to associate it with the name of
the bibliographic style so that it is automatically activated. Citation
styles can be defined for local \texttt{.bst} files by means of a
configuration file \thestyle\texttt{.cfg}.

It is compatible with the packages: \texttt{babel}, \texttt{index}, \texttt{citeref},
\texttt{showkeys}, \texttt{chapterbib}, \texttt{hyperref}, \texttt{koma} and
with the classes \texttt{amsbook} and \texttt{amsart}.
It can also emulate the sorting and compressing functions of the \texttt{cite} package
as well as the multiple citation (and merging) functions of Thorsten Ohl's \texttt{mcite} package.
(The \thestyle{} package, however, is not compatible with either \texttt{cite} or \texttt{mcite} themselves.)

Note that the \texttt{citeref} package (for adding citation page numbers in the
bibliography) must be loaded after \thestyle. (The \texttt{hyperref} package with
the option \texttt{pagebackref} also provides this feature, but with hyperlinks.)

The \thestyle\ package therefore acts as a single, flexible interface for
most of the available bibliographic styles.

\end{abstract}

\newpage\tableofcontents\newpage

\section{Introduction}

The \thestyle{} package is an extension to \LaTeX\ to allow author--year
citations along with numerical citations. Standard \LaTeX\ permits only
numerical, whereas all extensions for author--year prior to the release of
\thestyle{} in 1993 were limited to just that. Since they normally added new
commands (as \thestyle{} does  too), documents written with them could only be
used with numerical citations after extensive editing.

The \thestyle{} package has changed that; switching from author--year to
numerical citations is a matter of an option, with no alterations to the
source text. It has now become part of the standard \LaTeX\ installations,
and is supported (demanded) by many journals. It is the citation package of
choice by most of the \LaTeX\ community, mainly because of its flexibility
and configurability.

Like all packages, it is loaded in the document preamble, with possible options,
with, e.g.\
\begin{quote}
  |\usepackage[sectionbib,square]{natbib}|
\end{quote}
The option \texttt{sectionbib} specifies that, when used with the package
\texttt{chapterbib}, the bibliography will appear as a section at the end of
each chapter (Section~\ref{sec:chapbib}). The \texttt{square} option says
that references are to be enclosed in square bracket rather than round
parentheses. See Section~\ref{sec:opts} for a complete list of options.

The document text itself begins with, e.g.
\begin{quote}
\begin{verbatim}
\begin{document}
\bibliographystyle{plainnat}
\end{verbatim}
\end{quote}
which specifies \texttt{plainnat} to be the bibliography style used by the
\btx\ program that generates the actual bibliography from a database. The
style \texttt{plainnat} is the \thestyle{} version of the standard \texttt{plain}
(numerical only) style. See Section~\ref{sec:plainnat} for other styles, or
search the installation for \texttt{.bst} files.

The |\bibliographystyle| command can be given anywhere in the document, but
it makes sense to add it at the start where it can be easily identified (and
modified).

To make a citation in the text, use
\begin{quote}
  |\citep{jon90}| for a \emph{parenthetical} citation (Jones et al.,
  1990),\\
  |\citet{jon90}| for a \emph{textual} one, as Jones et al. (1990).
\end{quote}
Both |\citep| and |\citet| are defined by \thestyle{} and are thus not
standard. The standard \LaTeX\ command |\cite| should be avoided, because it
behaves like |\citet| for author--year citations, but like |\citep| for
numerical ones. There are many other commands for other special effects
(Section~\ref{sec:excite}).

In the above examples, \texttt{jon90} is the identifying key for the
reference, as found in the \btx\ database, or in the \texttt{thebibliography}
environment, Section~\ref{sec:bibitem}:
\begin{quote}
\begin{verbatim}
\begin{thebibliography}{1}
  \bibitem[Jones et al.(1990)]{jon90}
  . . . . .
\end{thebibliography}
\end{verbatim}
\end{quote}
This environment prints the actual bibliography, and the |\bibitem| commands
link the entries to the citations via the key, here \texttt{jon90}. The key
may be perfectly arbitrary as long as it is unique. The text in square
brackets contains the pieces of citation information, the authors \texttt{Jones et
al.} and the year \texttt{1990}. Note that these are two pieces of text that
may be packaged together in several different ways, depending on the citation
command. In fact, if numerical citations are selected, they are (almost)
ignored and only the sequence number is used as citation.

The \texttt{thebibliography} environment can be made by hand, but it is
better and safer to let \btx\ do it. For this, one needs the
|\bibliographystyle| command already mentioned, and near the end of the
document:
\begin{quote}
\begin{verbatim}
\bibliography{mybib}
\end{document}
\end{verbatim}
\end{quote}
Here \texttt{mybib} is the root name of the \btx\ database file
(\texttt{mybib.bib}) containing the data for the references needed in the
document.

The rest of this document presents all the gorey details about everything
possible with \thestyle.

\section{Using this Package}\label{sec:usage}


 In this paper, I distinguish between the citation \emph{mode} (author--year
 or numerical) and citation \emph{style} (the type of punctuation used for
 citations). The citation style is something that is independent of the
 bibliography style and is not programmed in the \texttt{.bst} files.

 \subsection{New Bibliography Styles}\label{sec:plainnat}

I provide three new \texttt{.bst} files to replace the standard \LaTeX\
numerical ones:
\begin{quote}\ttfamily
 plainnat.bst \qquad abbrvnat.bst \qquad unsrtnat.bst
\end{quote}

 These produce reference lists in the same style as the corresponding
 standard \texttt{.bst} file, but work with \thestyle{}. The
 advantage is that they can be used in both numerical and author--year
 mode.

 These \texttt{.bst} files are not meant to be exhaustive by any means. Other
 style files conforming to the \thestyle{} format exist, or may be generated
 with my \texttt{custom-bib} (also known as \texttt{makebst}) program.


 \subsection{The Syntax of the \texttt{thebibliography}}
 \label{sec:bibitem}

 The information on the cited author names and year are given as part of the
 |\bibitem| commands within the \texttt{thebibliography} environment. The
 \thestyle{} package expects that information to be in a certain format, which
 is maintained by the above bibliography styles. (It will also be able to
 interpret formats used by some earlier packages, such as \texttt{harvard} and
 \texttt{chicago}.) If one wishes to bypass \btx, one must make up the
 \texttt{thebibliography} oneself, such that it conforms to \thestyle.

 This syntax looks as follows:
 \begin{quote}
 |\bibitem[Jones et al.(1990)]{jon90}...|\\[1ex]
 or alternatively\\[1ex]
 |\bibitem[Jones et al.(1990)Jones, Baker, |\\
 \hspace*{2em}|and Williams]{jon90}...|
 \end{quote}
 The text in square brackets contains the pieces of citation texts, the short
 author list, \texttt{Jones et al.}, the year \texttt{1990}, and the optional long
 author list \texttt{Jones, Baker and Williams}. If the long list is missing,
 the short list will be used instead. The parentheses around the year are
 \emph{not} part of the text, but merely delimit the year from the author
 lists. Round parentheses must always be used, even if square brackets are
 wanted for the citations. And there must be no space before or after the year
 parentheses, else it will become part of the author list.

 \medskip\noindent
 \textbf{Note:} if any single |\bibitem| entry does not conform to a syntax that
 \thestyle{} understands, it switches stubbornly to numerical mode, since it
 otherwise has no idea what the author and year texts could be.

 \subsection{Basic Citation Commands}

 The \thestyle{} package can be used with bibliography styles that were
 intended for other, older packages, like \texttt{harvard}. However,
 the commands described in this and the next sections are defined by
 \thestyle{} and must be used even with those other bibliography styles.

 \DescribeMacro{\citet}
 \DescribeMacro{\citep}

The \thestyle\ package has two basic citation commands, |\citet| and
|\citep| for \emph{textual} and \emph{parenthetical} citations, respectively.
There also exist the starred versions |\citet*| and |\citep*| that print
the full author list, and not just the abbreviated one.
All of these may take one or two optional arguments to add some text before
and after the citation.
\begin{quote}
\begin{tabular}{l@{\quad$\Rightarrow$\quad}l}
  |\citet{jon90}| & Jones et al. (1990)\\
  |\citet[chap.~2]{jon90}| & Jones et al. (1990, chap.~2)\\[0.5ex]
  |\citep{jon90}| & (Jones et al., 1990)\\
  |\citep[chap.~2]{jon90}| & (Jones et al., 1990, chap.~2)\\
  |\citep[see][]{jon90}| & (see Jones et al., 1990)\\
  |\citep[see][chap.~2]{jon90}| & (see Jones et al., 1990, chap.~2)\\[0.5ex]
  |\citet*{jon90}| & Jones, Baker, and Williams (1990)\\
  |\citep*{jon90}| & (Jones, Baker, and Williams, 1990)
\end{tabular}
\end{quote}

 The starred versions can only list the full authors if the \texttt{.bst}
 file supports this feature; otherwise, the abbreviated list is printed.

 In standard \LaTeX, the |\cite| command can only take a single optional
 text for a note after the citation; here, a single optional text is a
 post-note, while two are the pre- and post-notes. To have only a pre-note, it
 is necessary to provide an empty post-note text, as shown above.

 More complex mixtures of text and citations can be generated with the
 all-purpose |\citetext| command in Section~\ref{sec:excite}.


Multiple citations may be made by including more than one
citation key in the |\cite| command argument.
 \textsl{If adjacent citations
 have the same author designation but different years, then the author
 names are not reprinted.}
\begin{quote}
\begin{tabular}{l@{\quad$\Rightarrow$\quad}l}
  |\citet{jon90,jam91}| & Jones et al. (1990); James et al. (1991)\\
  |\citep{jon90,jam91}| & (Jones et al., 1990; James et al. 1991)\\
  |\citep{jon90,jon91}| & (Jones et al., 1990, 1991)\\
  |\citep{jon90a,jon90b}| & (Jones et al., 1990a,b)
\end{tabular}
\end{quote}


These examples are for author--year citation mode. In numerical mode, the
results are different.
\begin{quote}
\begin{tabular}{l@{\quad$\Rightarrow$\quad}l}
  |\citet{jon90}| & Jones et al. [21]\\
  |\citet[chap.~2]{jon90}| & Jones et al. [21, chap.~2]\\[0.5ex]
  |\citep{jon90}| & [21]\\
  |\citep[chap.~2]{jon90}| & [21, chap.~2]\\
  |\citep[see][]{jon90}| & [see 21]\\
  |\citep[see][chap.~2]{jon90}| & [see 21, chap.~2]\\[0.5ex]
  |\citep{jon90a,jon90b}| & [21, 32]
\end{tabular}
\end{quote}

 The authors can only be listed if the \texttt{.bst} file supports
 author--year citations. The standard \texttt{.bst} files, such as
 \texttt{plain.bst} are numerical only and transfer no author--year
 information to \LaTeX. In this case, |\citet| prints ``\textbf{(author?)}
 [21].''

 \DescribeMacro{\cite}
 In the original versions of \thestyle, the traditional |\cite| command was
 used for both textual and parenthetical citations. The presence of an empty
 optional text in square brackets signalled parenthetical. This syntax has been
 retained for compatibility, but is no longer encouraged.

 This means that |\cite| (without notes) is the same as |\citet| in
 author--year mode, whereas in numerical mode, it is the same as |\citep|.
 The starred version, as well as the one or two optional notes, may also be
 used.

 It is possible to have multiple citations sorted into the same sequence as
 they appear in the list of references, regardless of their order as
 arguments to the |\cite| commands. The option \texttt{sort} is required for
 this feature. See Section~\ref{sec:sort}.

 Some publishers require that the first citation of any given reference be
 given with the full author list, but that all subsequent ones with the
 abbreviated list. Include the option \texttt{longnamesfirst} to enable this for
 \thestyle. See Section~\ref{sec:long}.

 \subsection{Extended Citation Commands}
 \label{sec:excite}

 \DescribeMacro{\citealt}
 \DescribeMacro{\citealp}
 \DescribeMacro{\citetext}
 \DescribeMacro{\citenum}

As an alternative form of citation, |\citealt| is the same as |\citet| but
\emph{without parentheses}. Similarly, |\citealp| is |\citep| without
parentheses.

The |\citenum| command prints the citation number, without parentheses, even
in author--year mode, and without raising it in superscript mode. This is
intended to be able to refer to citation numbers without superscripting them.

 Multiple references, notes, and the starred variants
 also exist for these, except for |\citenum|.

\begin{quote}
\begin{tabular}{l@{\quad$\Rightarrow$\quad}l}
  |\citealt{jon90}| & Jones et al.\ 1990\\
  |\citealt*{jon90}| & Jones, Baker, and Williams 1990\\
  |\citealp{jon90}| & Jones et al., 1990\\
  |\citealp*{jon90}| & Jones, Baker, and Williams, 1990\\
  |\citealp{jon90,jam91}| & Jones et al., 1990; James et al., 1991\\
  |\citealp[pg.~32]{jon90}| & Jones et al., 1990, pg.~32\\
  |\citenum{jon90}| & 11\\
  |\citetext{priv.\ comm.}| & (priv.\ comm.)
\end{tabular}
\end{quote}
The |\citetext| command
allows arbitrary text to be placed in the current citation parentheses.
This may be used in combination with |\citealp|.
 For example,
 \begin{verbatim}
 \citetext{see \citealp{jon90}, or even better \citealp{jam91}}
 \end{verbatim}
 to produce (see Jones et al., 1990, or even better James et al., 1991).



 \DescribeMacro{\citeauthor}
 \DescribeMacro{\citeyear}
 \DescribeMacro{\citeyearpar}
 \DescribeMacro{\citefullauthor}

In author--year schemes, it is sometimes desirable to be able to refer to
the authors without the year, or vice versa. This is provided with the
extra commands
\begin{quote}
\begin{tabular}{l@{\quad$\Rightarrow$\quad}l}
  |\citeauthor{jon90}| & Jones et al.\\
  |\citeauthor*{jon90}| & Jones, Baker, and Williams\\
  |\citeyear{jon90}|   & 1990\\
  |\citeyearpar{jon90}| & (1990)
\end{tabular}
\end{quote}

 There also exists a command |\citefullauthor| which is equivalent to
 |\citeauthor*|.

 If the full author information is missing, then |\citeauthor*| is
 the same as |\citeauthor|, printing only the abbreviated list.
 This also applies to the starred versions of |\citet| and |\citep|.

 If the author or year information is missing (as is the case with the
 standard \LaTeX{} \texttt{.bst} files), these commands issue a warning.

 \medskip\noindent
 \textbf{Note:}
 these commands may also be used with numerical
 citations, provided an author--year \texttt{.bst} file is being employed.

 \medskip\noindent
 \textbf{Note:} all |\cite..| commands have the same
 syntax, allowing multiple citations and up to two notes (there are, however,
 no starred |\citeyear| or |\citenum| variants). It does not really make much sense to add
 notes to |\citeyear| and |\citeauthor|, especially with multiple citations;
 however, this can be done, there will be no error message, but the results
 are sometimes strange. For example, in numerical mode, the notes are fully
 ignored, while in author--year mode, only the post-note is accepted.
 Multiple citations in |\citet| are also not recommended (nor are they in my
 opinion meaningful), but if they are used with notes, the pre-note will
 appear before each year, and the post-note only after the last year. These
 are admittedly bugs, but the effort to remove them is not justified by the
 questionable usefulness of these features.

 In summary, notes are only intended for |\citep| but they may also be used
 with |\citet| in author--year mode, with single citations. In any other
 situation, the results are unpredictable.

 \DescribeMacro{\Citet}
 \DescribeMacro{\Citep}
 \DescribeMacro{\Citealt}
 \DescribeMacro{\Citealp}
 \DescribeMacro{\Citeauthor}
 \subsection{Forcing Upper Cased Name}

If the first author's name contains a \textsl{von} part, such as ``della
Robbia'', then |\citet{dRob98}| produces ``della Robbia (1998)'', even at the
beginning of a sentence. One can force the first letter to be in upper case
with the command |\Citet| instead. Other upper case commands also exist.
\begin{quote}
\begin{tabular}{rl@{\quad$\Rightarrow$\quad}l}
  when & |\citet{dRob98}| & della Robbia (1998) \\
  then & |\Citet{dRob98}| & Della Robbia (1998) \\
     &   |\Citep{dRob98}| & (Della Robbia, 1998) \\
     &   |\Citealt{dRob98}| & Della Robbia 1998 \\
     &   |\Citealp{dRob98}| & Della Robbia, 1998 \\
     &   |\Citeauthor{dRob98}| & Della Robbia
\end{tabular}
\end{quote}
These commands also exist in starred versions for full author names.

 \medskip\noindent
 \textbf{Note:} the coding for the upper casing commands is tricky and likely
 buggy. It operates on the names that are stored in the |\bibitem| entry, and
 works even if old style font commands are used; however, \LaTeXe\ commands will
 cause it to crash. Thus\\
 \begin{tabular}[t]{ll}
   |\bibitem[{\it della Robbia}(1998)]{dRob98}| & is okay, but \\
   |\bibitem[\textit{della Robbia}(1998)]{dRob98}| & crashes.
 \end{tabular}
 ^^A I hope to improve this situation in future.


 \subsection{Citation Aliasing}
 \DescribeMacro{\defcitealias}
 \DescribeMacro{\citetalias}
 \DescribeMacro{\citepalias}

Sometimes one wants to refer to a reference with a special designation,
rather than by the authors, i.e. as Paper~I, Paper~II. Such aliases can be
defined and used, textual and/or parenthetical with:
\begin{quote}
\begin{tabular}{lcl}
  |\defcitealias{jon90}{Paper~I}|\\
  |\citetalias{jon90}| & $\Rightarrow$ & Paper~I\\
  |\citepalias{jon90}| & $\Rightarrow$ & (Paper~I)
\end{tabular}
\end{quote}
These citation commands function much like |\citet| and |\citep|: they may
take multiple keys in the argument, may contain notes, and are marked as
hyperlinks.


 A warning is issued if the alias is used before it is defined, or if an alias
 is redefined for a given citation. No warning is issued if an alias is
 defined for a citation key that does not exist; the warning comes when it is
 used!

 See Section~\ref{sec:yearless} for an alternative means of citing with
 a code name.

 \subsection{Authorless and Yearless References}\label{sec:yearless}
 What does one do about references that do not have authors? This has long
 bothered me but I do have a suggestion. Standard \btx\ styles make use of a
 \texttt{KEY} field in the entries to be used for alphabetizing when the
 authors or editors are missing. The author--year styles go even further and
 insert the \texttt{KEY} field in place of the authors. One can imagine giving
 a code designation for the work at this point. For example,
 \begin{quote}
 \begin{verbatim}
 @MANUAL{handbk98,
   title = {Assembling Computers},
   year = 1998,
   organization = {MacroHard Inc.},
   key = "MH-MAN"
 }
 \end{verbatim}
 \end{quote}
 With \texttt{plain}, the key text \texttt{MH-MAN} is used only to order the
 reference, but with \texttt{plainnat} and other author--year styles, it is
 used in place of the authors. One can then refer to it as
 |\citeauthor{handbk98}| to get MH-MAN or as |\citetext{\citeauthor{handbk98}}|
 for (MH-MAN), a parenthetical citation.

 This can be greatly simplified if
 the bibliography style leaves the date blank in the |\bibitem|
 entry, as
 \begin{quote}
 |\bibitem[MH-MAN()]{handbk98}|
 \end{quote}
 for then \thestyle\ suppresses the date, preceding punctuation, and the
 braces for |\citet|. This means that |\citet| and |\citep| behave
 automatically like the two examples above.
 The date still may appear in the text of the reference.

 The \thestyle\ bibliography styles have been modified accordingly to omit
 the date from the |\bibitem| entry when missing authors and/or editors
 are replaced by key text.

 Similarly, if the year is missing, it will be left blank in the |\bibitem|
 entry; thus citing such a work will only produce the authors' names.

 \medskip\noindent
 \textbf{Note:} there are many other possibilities with this feature. One can
 even produce citations like those of the \texttt{alpha} bibliography style,
 by placing the citation code in place of the authors in the |\bibitem| entry
 and leaving the year blank. A second code (or maybe even the authors
 themselves) could be placed where the full author list normally appears, to
 be printed with the starred version of the |\cite| commands. For example,
 \begin{quote}
 |\bibitem[MH-MAN()MacroHard Inc.]{handbk98}|
 \end{quote}

 \subsection{Extra Features in the \texttt{plainnat} Family}\label{sec:url}

 The special \texttt{.bst} files for \thestyle\ mentioned in
 Section~\ref{sec:plainnat} have a number of extra fields compared to the
 original files: \\
 \begin{tabular}{lp{10.5cm}}
 \texttt{ISBN} & for the ISBN number in books,\\
 \texttt{ISSN} & for the ISSN number in periodicals,\\
 \texttt{URL} & for the Internet address of online documents,\\
 \texttt{DOI} & the \emph{Digital Object Identifier} now being used by many
         journals as a more robust alternative to URL,\\
 \texttt{EID} & \emph{electronic ID}, a substitute for page numbers for online journals
         that also appear in print; also known as the \emph{sequence number}
         within the paper volume.
 \end{tabular}\\

 Both the DOI and URL tend to be very long, causing ugly line breaks or
 sticking out into the margin. This can be avoided by loading the \texttt{url}
 package by Donald Arseneau, which allows text to be broken at punctuation
 marks without a hyphen. This package is automatically detected by \thestyle{}
 and appropriate commands redefined. URLs are printed in typewriter font, DOI
 in roman. Without the \texttt{url} package, these numbers are never broken.

 As pointed out in Section~\ref{sec:yearless}, the \texttt{KEY} field is
 treated differently by \texttt{plainnat} than in \texttt{plain}. Whereas the
 latter uses this field only to alphabetize entries without authors,
 \texttt{plainnat} actually inserts it in place of the author, both in the
 reference text and in the citation label (|\bibitem| entries). Furthermore,
 the year is left empty in |\bibitem| so that |\citep| prints only the
 ``author'' text, which is now the \texttt{KEY}. This should be some code
 designation for the work.

 \subsection{Selecting Citation Punctuation}\label{sec:bibpunct}

 \DescribeMacro{\setcitestyle}
 The above examples have been printed with the default citation style.
 It is possible to change this, as well as to select numerical or
 author--year mode, by means of the |\setcitestyle| command, which takes as argument
 a comma-separated list of keywords. (This command is new to version~8.)

\begin{itemize}
\item
  Citation mode: |authoryear| or |numbers| or |super|
  (corresponds to fourth argument of |\bibpunct|).
  %^^A \ifNAT@numbers \ifNAT@super
\item
  Braces: |round| or |square| or |open={|\emph{char}|},close={|\emph{char}|}|
  (corresponds to first and second arguments of |\bibpunct|).
  %^^A \NAT@open \NAT@close
\item
  Between citations: |semicolon| or |comma| or |citesep={|\emph{char}|}|
  (corresponds to third argument of |\bibpunct|).
  %^^A \NAT@sep
\item
  Between author and year: |aysep={|\emph{char}|}|
  (corresponds to fifth argument of |\bibpunct|).
  %^^A \NAT@aysep
\item
  Between years with common author: |yysep={|\emph{char}|}|
  (corresponds to sixth argument of |\bibpunct|).
  %^^A \NAT@yrsep
\item
  Text before post-note: |notesep={|\emph{text}|}|
  (corresponds to optional argument of |\bibpunct|).
  %^^A \NAT@cmt
\end{itemize}
Defaults are |authoryear|, |round|, |comma|, |aysep={;}|, |yysep={,}|,
|notesep={, }|

Example~1, |\setcitestyle{square,aysep={},yysep={;}}| changes the author--year
output of
\begin{quote}
  |\citep{jon90,jon91,jam92}|
\end{quote}
into [Jones et al. 1990; 1991, James et al. 1992].

Example~2, |\setcitestyle{notesep={; },round,aysep={},yysep={;}}| changes the output of
\begin{quote}
  |\citep[and references therein]{jon90}|
\end{quote}
into (Jones et al. 1990; and references therein).

 \medskip\noindent
 \textbf{Note}:
 \begin{itemize}
   \item parameters not specified remain unchanged;
   \item the order of the keywords is unimportant;
   \item the punctuation between author and year applies only to author--year
     citations, not numerical;
   \item the |yysep| punctuation comes between years when multiple citations
     have the same, non-repeated authors; a space is always inserted
     as well; if the years too are same, the citation is printed as `2007a,b',
     without a space; to include a space, add it with |yysep={,~}|;
   \item for numerical citations with common authors, e.g,
     |\citet{jon90,jon91}| produces `Jones et al.\ [21, 22]' with the
     punctuation between the numbers; a space is automatically included for
     numbers, but not for superscripts.
   \item a single character does not really need to be in |{ }|,
      other than a comma; |yysep=;| is acceptable;
 \end{itemize}

 \DescribeMacro{\bibpunct}
 The older command for setting the citation style is |\bibpunct|, which
 takes one optional and six required arguments:
 \begin{enumerate}
 \item the opening bracket symbol, default = (
 \item the closing bracket symbol, default = )
 \item the punctuation between multiple citations, default = ;
 \item the letter `n' for numerical style, or `s' for numerical superscript
       style, any other letter for
       author--year, default = author--year;
 \item the punctuation that comes between the author names and the year
 \item the punctuation that comes between years or numbers when common author
       lists are suppressed (default = ,);
 \end{enumerate}

 The optional argument is the character preceding a post-note, default is a
 comma plus space. In redefining this character, one must include a space if
 one is wanted.

 The above |\setcitestyle| examples can be achieved with
 \begin{quote}
 |\bibpunct{[}{]}{,}{a}{}{;}| and\\
 |\bibpunct[,~]{(}{)}{,}{a}{}{;}|.
 \end{quote}


 \subsection{Predefining a citation style}\label{sec:predef}

 \DescribeMacro{\bibstyle@xxx}
 \DescribeMacro{\citestyle}

 If a particular set of citation punctuations is commonly used, it is possible
 to store it in the local \thestyle\texttt{.cfg} and to recall it with
 |\citestyle{|\emph{name}|}|. The
 definition is done by creating a command |\bibstyle@|\emph{name}, which sets
 the desired citation style.

 For example, the American Geophysical Union (AGU) demands in its publications
 that citations be made with square brackets and separated by semi-colons.
 There is an \texttt{agu.bst} file to accomplish most of the formatting, but
 such punctuations are not included in it. Instead, \thestyle{} has the
 definition
 \begin{quote}
 |\newcommand{\bibstyle@agu}{\bibpunct{[}{]}{;}{a}{,}{,~}}|
 \end{quote}
 which allows this set to be selected with the command
 \begin{quote}
 |\citestyle{agu}|
 \end{quote}

 There is an additional feature to such predefined styles: \thestyle\ attempts
 to execute |\citestyle| at the beginning of the document
 with the name of the bibliography style, as given by
 the |\bibliographystyle| command (and stored in the \texttt{.aux} file).
 This means that a citation style can be
 directly associated with a \texttt{.bst} file. Such implicit styles are
 immediately overwritten by any explicit style specifications, such as package
 options, or by |\setcitestyle|, |\bibpunct|, or |\citestyle| commands.

 Predefined citation styles are contained within the \thestyle\ code for the
 following bibliography styles:
 \begin{description}
   \item[\texttt{plain}] etc. (the 4 base styles): square braces, numerical, commas
   \item[\texttt{plainnat}] etc.: square braces, author--year, commas
   \item[\texttt{agu}] (American Geophysical Union): square, author--year, semi-colon
   \item[\texttt{egu}] (European Geosciences Union): round, author--year, semi-colon
   \item[\texttt{agms, dcu, kluwer}] (Harvard set): round, author--year
   \item[\texttt{cospar}] (Committe on Space Research): slashes, numerical, comma
   \item[\texttt{nature}] (Journal \textsl{Nature}): superscripts
 \end{description}
 There are others but they are mainly for my personal convenience. The above
 represent most of the major variations and can be used as required. The
 automatic association with other bibliography styles can only be achieved by
 putting the definitions into the local \thestyle\texttt{.cfg}.

 Note that the predefinitions for |plain| and |plainnat| specify square
 braces, thus changing the normal \thestyle\ default of round parentheses.

 The style defining commands may contain more than just |\bibpunct|
 or |\setcitestyle|.
 Some numerical citation scheme require even more changes. For example,
 \textsl{Nature} not only uses superscripted numbers for
 citations, it also prints the numbers in the list of references without
 the normal square brackets. To
 accommodate this, \thestyle{} contains the style definition
 \begin{quote}\begin{verbatim}
 \newcommand{\bibstyle@nature}%
     {\bibpunct{}{}{,}{s}{}{\textsuperscript{,}}%
      \renewcommand\bibnumfmt[1]{##1.}}
 \end{verbatim}
 \end{quote}
 The redefined |\bibnumfmt| command specifies how the reference numbers
 are to be formatted in the list of references itself.

 \subsection{Priority of Style Commands}\label{sec:priority}
 The citation style (punctuation and mode) can be selected by means of the
 |\setcitestyle|, |\bibpunct|, |\citestyle| commands or via
 |\bibliographystyle{|\textit{bst}|}| with a predefined |\bibstyle@|\textit{bst}.
 They can also be selected by package options
 (Section~\ref{sec:opts}). What happens if there are several conflicting
 selections?

 The lowest priority is assigned to the |\bibliographystyle|
 command, since this is implicit and not transparent to the user. The
 package options have the next priority. Finally, any selection by
 |\setcitestyle|,
 |\bibpunct| and/or |\citestyle| overrides those of the other methods.

 \subsection{Other Formatting Options}


 \DescribeMacro{\bibsection}
 The list of references normally appears as a |\section*| or |\chapter*|,
 depending on the main class. If one wants to redesign one's own heading,
 say as a numbered section with |\section|, then |\bibsection| may be
 redefined by the user accordingly.



 \DescribeMacro{\bibpreamble}
 A preamble appearing after the |\bibsection| heading may be inserted before
 the actual list of references by redefining |\bibpreamble|. This will appear in
 the normal text font unless it contains font declarations. The
 |\bibfont| applies to the list of references, not to this preamble.



 \DescribeMacro{\bibfont}
 The list of references is normally printed in the same font size and
 style as the main body. However, it is possible to redefine |\bibfont|
 to be font commands that are in effect within the \texttt{thebibliography}
 environment after any preamble. For example,
 \begin{quote}
 |\renewcommand{\bibfont}{\small}|
 \end{quote}


 \DescribeMacro{\citenumfont}
 Numerical citations may be printed in a different font. Define |\citenumfont|
 to be a font declaration like |\itshape| or even a command taking arguments
 like |\textit|.
 \begin{quote}
 |\renewcommand{\citenumfont}[1]{\textit{#1}}|
 \end{quote}
 The above is better than |\itshape| since it automatically adds italic
 correction.


 \DescribeMacro{\bibnumfmt}
 The format of the numerical listing in the reference list may also be changed
 from the default [32] by redefining |\bibnumfmt|, for example
 \begin{quote}
 |\renewcommand{\bibnumfmt}[1]{\textbf{#1}:}|
 \end{quote}
 to achieve \textbf{32}: instead.


 \DescribeMacro{\bibhang}
 The list of references for author--year styles uses a hanging indentation
 format: the first line of each reference is flush left, the following lines
 are set with an indentation from the left margin. This indentation is 1~em
 by default but may be changed by redefining (with |\setlength|) the
 length parameter |\bibhang|.


 \DescribeMacro{\bibsep}
 The vertical spacing between references in the list, whether author--year
 or numerical, is controlled by the length |\bibsep|. If this is set to
 0~pt, there is no extra line spacing between references. The default
 spacing depends on the font size selected in |\documentclass|, and is
 almost a full blank line. Change this by redefining |\bibsep| with
 |\setlength| command.



 \subsection{Automatic Indexing of Citations}

 \DescribeMacro{\citeindextrue}
 \DescribeMacro{\citeindexfalse}

If one wishes to have the citations entered in the \texttt{.idx} indexing
file, it is only necessary to issue |\citeindextrue| at any point in the
document. All following |\cite| commands, of all variations, then insert
the corresponding entry to that file. With |\citeindexfalse|, these
entries will no longer be made.


 The |\bibitem| commands in the \texttt{thebibliography} environment will
 also make index entries. If this is not desired, then issue
 |\citeindexfalse| before |\bibliography| or |\begin{thebibliography}|.

 Of course, |\makeindex| must also be issued in the preamble to activate
 indexing, as usual. Otherwise, no indexing is done at all.

 Make sure that the document has been processed at least twice after the
 last \btx\ run before running the \texttt{makeindex} program.

 \DescribeMacro{\NAT@idxtxt}
 The form of the index entries is set by the internal |\NAT@idxtxt|, which
 can be redefined by hackers if wanted (in the \thestyle\texttt{.cfg} file
 please). By default, it prints the short author list plus date or number in the
 current parenthesis style and citation mode.

 The \thestyle\ package can also be used with the \texttt{index} package of
 David~M. Jones. The order in which the packages are loaded is not
 important.

 \DescribeMacro{\citeindextype}
 In that package, multiple index lists may be made by means of a |\newindex|
 command. For example, it may be desirable to put all the citation indexing
 into a separate list. First that list must be initiated with, e.g.,
 \begin{quote}
    |\newindex{cite}{ctx}{cnd}{List of Citations}|
 \end{quote}
 and then the automatic citation indexing associated with this list with the
 \thestyle\ command
 \begin{quote}
    |\renewcommand{\citeindextype}{cite}|
 \end{quote}
 See the documentation for \texttt{index.sty} for details.

 \subsection{Hyper\TeX\ Compatibility}
 The \thestyle\ package is compatible with the \texttt{hyperref}
 package of Sebastian Rahtz and Heiko Oberdiek.
 , for use with
 \LaTeX\ $\rightarrow$ HTML conversions, pdf\TeX, \texttt{pdfmark}.
 The compatibility is of a mutual nature: both packages contain
 coding that interact with that of the other.

 There is a special option to the \texttt{hyperref} package: \texttt{breaklinks}
 which allows the link text to be split over to the next
 line, avoiding many overfull line messages. By default, the link text is
 restricted to be on one line. For numerical citations, this may be desirable,
 but for author--year ones with long link texts, it is not.

 \subsection{Multiple Bibliographies in One Document}\label{sec:chapbib}


The \thestyle\ package is compatible with the \texttt{chapterbib} package
 of Donald Arseneau and Niel Kempson,
which makes it possible to have several bibliographies in one document.
 The usual application is to have bibliographies in each chapter of a book,
 especially if they have been written by different authors.

 The \texttt{chapterbib} package works in a very natural way for the
 author; only the editor who puts all the chapters together into one book
 has to do some extra work.

The package makes use of the |\include| command, and each |\include|d file
has its own bibliography.
 For large books, it makes very good sense to take advantage of this feature
 in any case.

 To review the use of |\include|, recall that the main file
 \begin{verbatim}
   \documentclass{...}
   \includeonly{ch2}
   \begin{document}
     \include{ch1}
     \include{ch2}
     \include{ch3}
   \end{document}
 \end{verbatim}
 will process only the file \texttt{ch2.tex} as though the files
 \texttt{ch1.tex} and \texttt{ch3.tex} were also present. That is, all
 counters, especially the page and section numbers, as well as
 cross-referencing definitions, will function as if the whole document
 were processed. The trick is that each |\include|d file has it own
 \texttt{.aux} file containing these definitions, and they are all read
 in every time, even if the corresponding \texttt{.tex} file is not. The
 \texttt{.aux} files also contain the citation information for \btx,
 something that the \texttt{chapterbib} package exploits.

 If |\usepackage{chapterbib}| has been given, the keys in each |\cite|
 and |\bibitem| command are associated with the current |\include|d file
 and are distinguished from the identical key in a different file. Each of
 these files must contain its own |\bibliography| and |\bibliographystyle|
 commands. One processes \btx\ on each file separately before processing
 it under \LaTeX\ (at least twice).

 \subsubsection{Special Considerations for \thestyle\ and
           \texttt{chapterbib}}

The order in which the \texttt{chapterbib} and \thestyle\ packages are loaded
is unimportant.

The \texttt{chapterbib} package provides an option \texttt{sectionbib}
that puts the bibliography in a |\section*| instead of |\chapter*|,
something that makes sense if there is a bibliography in each chapter.
This option will not work when \thestyle\ is also loaded; instead, add
the option to \thestyle.
 (The \texttt{sectionbib} option can always be
 given, but it only has meaning for the \texttt{book} and \texttt{report}
 classes, or for classes derived from them.)

Every |\include|d file must contain its own
|\bibliography| command where the bibliography is to appear. The database
files listed as arguments to this command can be different in each file,
of course. However, what is not so obvious, is that each file must also
contain a |\bibliographystyle| command, with possibly differing arguments.

As of version~8.0, the citation style, including mode (author--year or
numerical) may also differ between chapters. The |\setcitestyle| command
 as well as |\bibpunct| and |\citestyle|
can be issued at any point in the document, in particular in different
chapters.
(And this is the only time it would make sense to do so.)



 \subsection{Sorting and Compressing Numerical Citations}
 \label{sec:sort}

 Another package by Donald Arseneau, \texttt{cite.sty}, reimplements the
 entire (numerical) citation system such that one can control the
 punctuation and citation format, all of which is done by \thestyle\ as
 well. However, it also can sort and compress numerical citations,
 something that is required by some journals.

 What this means is that when multiple citations are given with a single
 |\cite| command, the normal order of the numbers is in the sequence
 given. This is usually a wild list of numbers, such as [4,2,8,3]. With
 the \texttt{cite} package, this list becomes [2--4,8].

 It is impossible to make the \texttt{cite} and \thestyle\ packages
 compatible, since both reimplement |\cite| from scratch. Instead, I have
 taken over some of the coding from \texttt{cite.sty}, modifying it for
 \thestyle{}. This coding is activated by including one of the options
 \texttt{sort} or \texttt{sort\&compress} in the |\usepackage| command.

 For author--year citations, the option \texttt{sort} orders the citations in
 a single |\citep| or |\citet| command into the sequence in which they appear
 in the list of references. This is normally alphabetical first, year second.
 This should avoid citations of the type: ``James et~al.\ (1994b,a)''. For
 author--year mode, the \texttt{sort\&compress} option is identical to
 \texttt{sort}.



 \subsection{Merged Numerical References}
 \label{sec:merge}


 \begin{quote}
 \textbf{Note:} the |merge| coding was contributed by
 Arthur Ogawa for the American Physical Society.
 \end{quote}

 Thorsten Ohl's \texttt{mcite} package cannot be used together with
 \thestyle. Instead, one must invoke the package option
 \texttt{merge} with the |\usepackage| statement.


With this option in effect, citation keys within a multiple |\citep| command
may contain a leading * that causes them to be merged in the bibliography
together with the previous citation as a single entry with a single reference
number. For example, |\citep{feynmann,*salam,*epr}| produces a single number,
and all three references are listed in the bibliography under one entry with
that number.

 Certain restrictions of Thorsten Ohl's \texttt{mcite} package are lifted: a *
 prefix can be used with any citation key, and Ohl's restrictions on syntax do
 not apply.

 An additional feature allows text to be inserted inside the bibliography entry:
 within the argument of \verb+\citep+, a cite key may be preceded by two optional arguements, e.g.,
 \verb+\citep{[+\emph{pre}\verb+][+\emph{post}\verb+]+\emph{key}\verb+}+,
 where \emph{pre} represents text to be prepended to the reference (\verb+\bibitem+ content),
 with \emph{post} appended thereto.

 Thus it is possible for the user to mark up the document:
 \begin{verbatim}
 text \citep{*[{See, e.g., }][ for a simpler explanation]ablebaker}
 \end{verbatim}
 The above example illustrates the use of the star-form and both optional arguments.

 \medskip\noindent
 \textbf{Caution:} Because the comma (,) character is part of the syntax of \verb+\citep+,
 and because the bracket delimiters ([]) are mere text characters and not true delimiters (like the braces),
 you must enclose the optional argument in braces (\verb+{}+) if a comma is present in the
 argument (that goes for brackets, too, of course).

The \texttt{elide} option also activates the merging features, but also sees
to it that common parts of the merged references (e.g., authors) are not
repeated but are written only once in the single bibliography entry.

The \texttt{mcite} option turns off the merging and eliding features, but
allows the special syntax (the * and optional inserted texts) to be ignored.

These functions are available only to numerical-mode citations, and only when
used parenthetically, similar to the restrictions on \texttt{sort} and
\texttt{compress}.

They also require special \texttt{.bst} files, as provided for example by the
American Physical Society for their REV\TeX\ class.


 \subsection{Long Author List on First Citation}\label{sec:long}

 A feature that has often been requested by otherwise happy users of
 \thestyle\ is one that is found in the \texttt{harvard} package as
 standard: with the first citation of any reference, the full author list is
 printed, and afterwards only the abbreviated list. One can control this
 with |\citet*| for the first citation, and |\citet| or |\citep| thereafter.
 However, the automatic feature is very desired.

 This can be activated with the option
 \texttt{longnamesfirst}.



 \DescribeMacro{\shortcites}
 Some references have so many authors that you want to suppress the
 automatic long list only for them. In this case, issue
 \begin{quote}
   |\shortcites|\marg{key-list}
 \end{quote}
 before the first citations, and those included in \emph{key-list} will have
 a short list on their first citation.

 Full author lists can still be forced at any time with the starred
 variants.

 \section{Numerical Citations with Author--Year Styles}\label{sec:6.0}

 It is possible to produce numerical citations with any author-year
 \texttt{.bst} file, with minimal change to the text. The commands |\citet|
 and |\citep| will produce sensible results in both modes, without any special
 editing. Obviously, the opposite is not possible; a \texttt{.bst} file
 intended for numerical citation can never produce author--year citations,
 simply because the information is not transferred to the auxiliary file.

 \subsection{Selecting Numerical Mode}
 By default, \thestyle{} is in author--year mode. This can be changed by
 \begin{enumerate}
   \item selecting a numerical bibliography style with predefined
         citation style, defined either in the package or in the local
         configuration file;

   \item giving options \texttt{numbers} or \texttt{super} to the
         |\usepackage| command;

   \item issuing |\setcitestyle{numbers}|;

   \item issuing |\citestyle| with the name of a predefined numerical
         bibliography style (like \texttt{plain} and use it with \texttt{plainnat}).
 \end{enumerate}

 The \thestyle{} package will automatically switch to numerical mode if
 any one of the |\bibitem| entries fails to conform to the possible
 author--year formats. There is no way to override this, since such an
 entry would cause trouble in the author--year mode.

 There are certain special `numerical' styles, like that of the standard
 \texttt{alpha.bst}, which include a non-numerical label in place of the
 number, in the form
 \begin{quote} |\bibitem[ABC95]{able95}| \end{quote}
 As far as \thestyle\ is concerned, this label does not conform to the
 author--year possibilities and is therefore considered to be numerical.
 The citation mode switches to numerical, and |\cite{able95}| prints
 [ABC95].

 See however, the end of Section~\ref{sec:yearless} for another possibility.
 The above result can be achieved with
 \begin{quote} |\bibitem[ABC95()]{able95}| \end{quote}

 \section{Local Configuration}
 It is possible to add a local configuration file
 \thestyle\texttt{.cfg}, which is read in, if it exists, at
 the end of the package. It may thus contain coding to supecede that in
 the package, although its main purpose is to allow the user to add his or her
 own |\bibstyle@|\textit{bst} definitions to couple citation punctuation
 with local bibliography styles or for use with |\citestyle|.



 \section{Package Options}\label{sec:opts}
 When a package is loaded with |\usepackage|, one can add options to select
 different features, as
 \begin{quote}
 |\usepackage[|\emph{options}|]{|\thestyle|}|
 \end{quote}

 The options available provide another means of specifying the
 punctuation for citations:
\begin{description}
\item[\ttfamily round] (default) for round parentheses;
\item[\ttfamily square] for square brackets;
\item[\ttfamily curly] for curly braces;
\item[\ttfamily angle] for angle brackets;
\item[\ttfamily semicolon] (default) to separate multiple citations with
     semi-colons;
\item[\ttfamily colon] the same as \texttt{semicolon}, an earlier mistake in
     terminology;
\item[\ttfamily comma] to use commas as separators;
\item[\ttfamily authoryear] (default) for author--year citations;
\item[\ttfamily numbers] for numerical citations;
\item[\ttfamily super] for superscripted numerical citations, as in
     \textsl{Nature};
\item[\ttfamily sort] orders multiple citations into the sequence in
     which they appear in the list of references;
\item[\ttfamily sort\&compress] as \texttt{sort} but in addition multiple
     numerical citations are compressed if possible (as 3--6, 15);
\item[\ttfamily compress] to compress without sorting, so compression only
     occurs when the given citations would produce an ascending sequence of
     numbers;
\item[\ttfamily longnamesfirst] makes the first citation of any reference
     the equivalent of the starred variant (full author list) and subsequent
     citations normal (abbreviated list);
\item[\ttfamily sectionbib] redefines |\thebibliography| to issue
     |\section*| instead of |\chapter*|; valid only for classes with a
     |\chapter| command; to be used with the \texttt{chapterbib} package;
\item[\ttfamily nonamebreak] keeps all the authors' names in a citation on
     one line; causes overfull hboxes but helps with some \texttt{hyperref}
     problems;
\item[\ttfamily merge] to allow the * prefix to the citation key,
     and to merge such a citation's reference with that of the previous
     citation;
\item[\ttfamily elide] to elide common elements of merged references, like
     the authors or year;
\item[\ttfamily mcite] to recognize (and ignore) the merging syntax.
\end{description}
 The options \texttt{curly} and \texttt{angle} are not really serious; I only
 added them because that completes the normal list of bracket types. The only
 other citation possibilities I know have really encountered are solidus, e.g. /21/,
 or something like (Ref. 21). These must be set with |\setcitestyle{open={/},close={/}}|.


 The package options are overwritten by any explicit |\setcitestyle|, |\bibpunct|,
 or |\citestyle| commands. And both these commands and the package options
 turn off the automatic setting with |\bibliographystyle|, if effective.

 \section{Reference Sheet}

 A summary of the main points on using \thestyle\ can be obtained by
 \LaTeX{}ing the file \texttt{natnotes.tex}, which is extracted from the main
 source file \thestyle\texttt{.tex} with the \texttt{docstrip} option
 \texttt{notes}. This is intended to act as a handy reference sheet.

 This file should be extracted automatically by the supplied installation file,
 \thestyle\texttt{.ins}.

^^A The following is a summary that goes into the .sty file
^^A It is not printed in the documentation, since the Reference Sheet exists

 \section{Options with \texttt{docstrip}}
 The source \texttt{.tex} file is meant to be processed with
 \texttt{docstrip}, for which a number of options are available:
 \begin{description}
 \item[\ttfamily all] includes all of the other interfaces;

 \item[\ttfamily apalike] allows interpretation of minimal \texttt{apalike}
    form of |\bibitem|;

 \item[\ttfamily newapa] allows |\citeauthoryear| to be in the optional
    argument of |\bibitem| along with the punctuation commands of
    \texttt{newapa.sty};

 \item[\ttfamily chicago] is the same as \texttt{newapa};

 \item[\ttfamily harvard] includes interpretation of |\harvarditem|;

 \item[\ttfamily astron] allows |\astroncite| to appear in the optional
    argument of |\bibitem|;

 \item[\ttfamily authordate] adds the syntax of the |\citename| command.

 \end{description}

 The remaining options are:
 \begin{description}
 \item[\ttfamily package] to produce a \texttt{.sty} package file with most
     comments removed;

 \item[\ttfamily notes] extracts a summary of usage to be used as a
     reference sheet; the resulting file is to be \LaTeX{}ed;

 \item[\ttfamily driver] to produce a driver \texttt{.ltx} file that will
     print out the documentation when \LaTeX'd. This file can be modified
     to produce various alternatives (page size, fonts, manual only, or with
     annotated code). The \thestyle\texttt{.tex} file is itself such a driver
     but it should never, ever be edited by a user.

 \end{description}
 The source file \texttt{\filename.tex} is itself a driver file and can
 be processed directly by \LaTeX.


 \section{Other Author--Year Solutions}
 \begin{quote}\slshape
 This section is of historical interest only.
 \end{quote}

 Before \thestyle{} was published in 1993, there were several other attempts
 to provide author--year citations, some of which inspired \thestyle. A few of
 these are still maintained and used, and for that reason, \thestyle{} has
 attempted to include their |\bibitem| syntaxes, to be compatible with those
 \texttt{.bst} files.

 Most of these `packages' are really \LaTeX~2.09 style files, so do not have
 features available with the modern \LaTeXe.

 \subsection{The \texttt{natsci.bst} Style}
 What gave me my first inspiration was Stephen Gildea's \texttt{natsci.bst}
 for use with his \texttt{agujgr.sty} file. This showed me that the problem
 was solvable. However, Gildea's formats |\bibitem| with just
 the abbreviated
 authors and year. Thus only parenthetical citations can be accommodated.

 The name \texttt{natsci} stands for \emph{natural sciences}, and it was this
 that led to the name \thestyle. (This is admittedly an ugly name, but it is
 now established and cannot be changed so easily.)

 \subsection{The \texttt{apalike.bst} Style}
 Oren Patashnik, the originator of \btx{} and the standard \texttt{.bst}
 files, has also worked on an author--year style, called \texttt{apalike.bst}
 with a corresponding \texttt{apalike.sty} to support it. Again, only the
 parenthetical citation is provided. Its functionality is identical to that
 of the \texttt{natsci} files.

 The form of the \texttt{thebibliography} entries in this system is
 \begin{quote}
 |\bibitem[Jones et al., 1990]{jon90}...|
 \end{quote}
 This is the most minimal form that can
 be given. I name it the \texttt{apalike} variant, after Patashnik's
 \texttt{apalike.bst} and \texttt{apalike.sty}. However, there could be many
 independent \texttt{.bst} files that follow this line, such as the
 \texttt{natsci} styles.

 The bibliography style files belonging to this group include:
 \begin{quote}
 \texttt{apalike}, \texttt{apalike2}, \texttt{cea}, \texttt{cell},
 \texttt{jmb}, \texttt{phapalik}, \texttt{phppcf}, \texttt{phrmp}
 \end{quote}

 \subsection{The \texttt{newapa} Style}
 A major improvement was achieved with \texttt{newapa.bst} and the
 accompanying \texttt{newapa.sty} files by Stephen N. Spencer and Young U.
 Ryu. Under their system, three separate items of information are included
 in the |\bibitem| label, to be used as required. These are: the full
 author list, the abbreviated list, and the year. This is accomplished by
 means of a |\citeauthoryear| command included in the label, as
 \begin{quote}
 |\bibitem[\protect\citeauthoryear{Jones, Barker,|\\
 |  and Williams}{Jones et al.}{1990}]{jon90}...|
 \end{quote}
 Actually, this only illustrates the basic structure of |\citeauthoryear|;
 the \texttt{newapa} files go even further to replace some words and
 punctuation
 with commands. For example, the word `and' above is really
 |\betweenauthors|, something that must be defined in the \texttt{.sty} file.
 Of course, |\citeauthoryear| is also defined in that file. A
 number of different |\cite| commands are available to print out the
 citation with complete author list, with the short list, with or without
 the date, the textual or parenthetical form.

 Thus the |\citeauthoryear| entry in |\bibitem| is very flexible,
 permitting the style file to generate every citation form that one might
 want. It is used by a number of other styles, with corresponding
 \texttt{.sty} files. They all appear to have been inspired by
 \texttt{newapa.bst}, although they lack the extra punctuation commands.

 Bibliographic style files belonging to the \texttt{newapa} group include
 \begin{quote}
 \texttt{newapa}, \texttt{chicago}, \texttt{chicagoa}, \texttt{jas99},
 \texttt{named}
 \end{quote}

 \noindent
 \textbf{Note:} the last of these, \texttt{named.bst}, uses |\citeauthoryear| in a
 slightly different manner, with only two arguments: the short list and
 year.

 \subsection{The Harvard Family}
 The same effect is achieved by a different approach in the Harvard family
 of bibliographic styles. Here a substitute for |\bibitem| is used, as
 \begin{quote}
 |\harvarditem[Jones et al.]{Jones, Baker, and|\\
 |   Williams}{1990}{jon90}...|
 \end{quote}
 The accompanying interface package file is called \texttt{harvard.sty}
 and is written by Peter Williams and Thorsten Schnier. It
 defines |\harvarditem| as well as the citation commands |\cite|, for
 parenthentical, and |\citeasnoun|, for textual citations. The first
 citation uses the long author list, following ones the shorter list, if
 it has been given in the optional argument to |\harvarditem|.

 Bibliography styles belonging to the Harvard family are
 \begin{quote}
 \texttt{agsm}, \texttt{dcu}, \texttt{kluwer}
 \end{quote}

 This package has been updated for \LaTeXe, with many additions to
 add flexibility. The result is a powerful interface that should meet most
 citation needs. (It does not suppress repeated authors, though,
 as \thestyle{} does.)

 \subsection{The Astronomy Style}
 Apparently realizing the limitations of his \texttt{apalike} system, Oren
 Patashnik went on to develop a `true' \texttt{apa} bibliographic style,
 making use of the method already employed by an astronomy journal. This
 is actually very similar to the \texttt{newapa} label but with only the
 short list of authors:
 \begin{quote}
 |\bibitem[\protect\astroncite{Jones et al.}{1990}]{jon90}|\\
 |   ...|
 \end{quote}
 It requires the package file \texttt{astron.sty}
 or any other style that defines |\astroncite| appropriately.

 Bibliographic styles belonging to the astronomy group are
 \begin{quote}
 \texttt{apa}, \texttt{astron}, \texttt{bbs}, \texttt{cbe},
 \texttt{humanbio}, \texttt{humannat}, \texttt{jtb}
 \end{quote}

 This is as good as the |\citeauthoryear| command, although not as
 flexible since the full list of authors is missing.

 \subsection{The \texttt{authordate} Style}
 Finally, I also found some packages making use of a label command
 called |\citename| in the form
 \begin{quote}
 |\bibitem[\protect\citename{Jones et al., }1990]{jon90}|\\
 |    ...|
 \end{quote}

 This is not a good system since the author list and date are not cleanly
 separated as individual arguments, and since the punctuation is included
 in the label text. It is better to keep the punctuation fully removed, as
 part of the definitions in the \texttt{.sty} file, for complete flexibility.

 Bibliographic styles belonging to this group are
 \begin{quote}
 \texttt{authordate1}, \texttt{authordate2}, \texttt{authordate3},
 \texttt{authordate4}, \texttt{aaai-named}
 \end{quote}
 with accompanying style file \texttt{authordate1-4.sty}.
